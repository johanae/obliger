\documentclass[11pt]{amsart}
%\usepackage{pstricks,pst-plot}
\usepackage{tikz}
\usepackage[utf8]{inputenc}

\newtheorem{theorem}{Theorem}[section]
\newtheorem{lemma}[theorem]{Lemma}
\newtheorem{corollary}[theorem]{Corollary}
%\newtheorem{proposition}[theorem]{Proposition}
\newtheorem{prop}[theorem]{Proposition}
\newtheorem{non}[theorem]{Nonsense}
\newtheorem{observation}[theorem]{Observation}

\theoremstyle{definition}
\newtheorem{definition}[theorem]{Definition}
\newtheorem{example}[theorem]{Example}
\newtheorem{xca}[theorem]{Exercise}
\newtheorem{problem}[theorem]{Problem}
\newtheorem{remark}[theorem]{Remark}
\newtheorem{nremark}[theorem]{Bemerkning}
\newtheorem{question}[theorem]{Question}
\newtheorem{conjecture}[theorem]{Conjecture}
\newtheorem{improvement}[theorem]{Improvement}
\newtheorem{discus}[theorem]{Discus}
\newtheorem{ptheorem}[theorem]{Possible Theorem}
\newtheorem{project}[theorem]{Project}

\newcommand\hra{\hookrightarrow}

\newcommand{\cA}{\mathcal{A}}
\newcommand{\cB}{\mathcal{B}}
\newcommand{\cR}{\mathcal{R}}
\newcommand{\cC}{\mathcal{C}}
\newcommand{\cD}{\mathcal{D}}
\newcommand{\cE}{\mathcal{E}}
\newcommand{\cF}{\mathcal{F}}
\newcommand{\cG}{\mathcal{G}}
\newcommand{\cH}{\mathcal{H}}
\newcommand{\cI}{\mathcal{I}}
\newcommand{\cJ}{\mathcal{J}}
\newcommand{\cK}{\mathcal{K}}
\newcommand{\cL}{\mathcal{L}}
\newcommand{\cN}{\mathcal{N}}
\newcommand{\cO}{\mathcal{O}}
\newcommand{\cS}{\mathcal{S}}
\newcommand{\cZ}{\mathcal{Z}}
\newcommand{\cP}{\mathcal{P}}
\newcommand{\cT}{\mathcal{T}}
\newcommand{\cU}{\mathcal{U}}
\newcommand{\cV}{\mathcal{V}}
\newcommand{\cX}{\mathcal{X}}
\newcommand{\cW}{\mathcal{W}}


\newcommand{\A}{\mathbb{A}}
\newcommand{\B}{\mathbb{B}}
\newcommand{\C}{\mathbb{C}}
\newcommand{\D}{\mathbb{D}}
\renewcommand{\H}{\mathbb{H}}
\newcommand{\N}{\mathbb{N}}
\newcommand{\Q}{\mathbb{Q}}
\newcommand{\Z}{\mathbb{Z}}
\renewcommand{\P}{\mathbb{P}}
\newcommand{\R}{\mathbb{R}}
\newcommand{\U}{\mathbb{U}}
\newcommand{\bT}{\mathbb{T}}
\newcommand{\bD}{\mathbb{D}}

\newcommand\Subset{\subset\subset}
\newcommand\wt{\widetilde}
\newcommand\ddt{\frac{d}{dt}}


\newcommand{\Aut}{\mathop{{\rm Aut}}}
\newcommand{\seq}[1]{\{#1\}_{n\in\N}}

\def\di{\partial}
\def\bs{\backslash}
\def\e{\epsilon}


\setlength\parindent{0pt} % Removes all indentation from paragraphs

\numberwithin{equation}{section}


%
%
%  THE DOCUMENT
%
%

\begin{document}
\title[MAT-INF4130]{Compulsory exercise 1}

\author{Johan Åmdal Eliassen}

\maketitle
\subjclass[Mat-inf4130 2015]{}

\date{\today}
\keywords{}

\section{Problem 1}
Let $A \in \C^{n\times n}$ be a nonsingular triangular matrix with inverse $B = [\mathbf{b_1}\: \mathbf{b_2}\: \cdots \:\mathbf{b_n}]$.

Then we have
\begin{equation}
AB =
[
A \mathbf{b}_1\: A \mathbf{b}_2 \_\cdots \:A\mathbf{b}_n
]\ = I\,,
\end{equation}
that is,
\begin{equation}
A\mathbf{b}_k = \mathbf{e}_k\,,
\end{equation}
$\mathbf{e_k} \in \R^n$ denoting the $k$'th unit vector.

As usual, denote entry $(i,j)$ of a matrix $A$ by $a_{i,j}$.

\subsection*{Case 1: A lower triangular}
For $k = 1$, there is nothing to prove. 
Let $1 < k \leq n$. Define $A_{-[n-k+1]} \in \C^{n-k+1 \times n - k +1}$ by
\begin{equation}
A_{-[n-k+1]}
:= 
\left(
\begin{matrix}
a_{k,k} &\cdots &a_{k, n} \\
\vdots &\vdots & \vdots \\
a_{n,k} & \cdots &a_{n,n}
\end{matrix}
\right)\,.
\end{equation}
Then, for some matrix $L \in \C^{k-1 \times n-k+1}$, we have
\begin{equation}
A \mathbf{b}_k =
\left(
\begin{matrix}
A_{[k-1]} & 0 \\
L & A_{-[n-k+1]}
\end{matrix}
\right)
\left(
\begin{matrix}
\mathbf{b}_k^u \\
\mathbf{b}_k^l
\end{matrix}
\right)
= 
\left(
\begin{matrix}
A_{[k-1]}\mathbf{b}_k^u \\
L\mathbf{b}_k^u + A_{-[n-k+1]}\mathbf{b}_k^l
\end{matrix}
\right)
=
\left(
\begin{matrix}
0 \\ \mathbf{e}_1
\end{matrix}
\right)\,,
\end{equation}
where $\mathbf{b}_k^u \in \R^{k-1}$ denote the $k-1$ upper elements of $\mathbf{b}_k$ and $\mathbf{b}_k^l \in \R^{n-k+1}$ denote the remaining $n-k+1$ lower elements.

Since $A$ is nonsingular and triangular, so is $A_{[k]}$ (because necessarily the diagonal elements of $A_{[k]}$ are then all nonzero), so we conclude that $\mathbf{b}_k^u \equiv 0$. Thus, we are left only with the system of equations
\begin{equation}
A_{-[n-k+1]}\mathbf{b}_k^l = \mathbf{e_1}\,,
\end{equation}
or in MATLAB notation,
$A(k:n, k:n)b(k:n,k:n) = I(k:n,k)$, as was to be shown.
\qed
\subsection*{Case 2: A upper triangular}
Similarly, for some matrix $U$, we write
\begin{equation}
A \mathbf{b}_k= 
\left(
\begin{matrix}
A_{[k]} & U \\
0 & A_{-[n-k]}
\end{matrix}
\right)
\left(
\begin{matrix}
\mathbf{b}_k^u \\
\mathbf{b}_k^l
\end{matrix}
\right)
=
\left(
\begin{matrix}
A_{[k]}\mathbf{b}_k^u + U\mathbf{b}_k^l \\
A_{-[n-k]}\mathbf{b}_k^l
\end{matrix}
\right)
\,,
\end{equation}
leading us to conclude that $b_k^l \equiv \mathbf{0}$, so we need only solve
\begin{equation}
A_{[k]}\mathbf{b}_k^u = \mathbf{e}_k\,,
\end{equation}
or in MATLAB notation,
$A(1:k, 1:k)\mathbf{b}_k(1:k) = I(1:k,k)$.
\qed

As a final remark, we note that the nondiagonal nonzero elements of $B$ may be stored in place of the zero elements of $A$. That is, we may simply store $b_{i,j}$ as $a_{j,i}$, for all elements of $B$ not on the diagonal or on the zero part of $B$.

The entries on the diagonal are not worth storing, as $b_{i,i} = 1/(a_{i,i})$.

\section*{Problem 2}
We have, for $j \geq k = 1, 2, \cdots, n$, if $A$ is lower triangular,
\begin{equation}
A_{-[n-k+1]}\mathbf{b}_k^l = \mathbf{e_1}\,
\end{equation}
which is solved by
\begin{align*}
(\mathbf{b}_k^l)_1 = b_{k,k} &= \frac{-1}{a_{k,k}}\,, \:\: \text{1 division}\,, \\
(\mathbf{b}_k^l)_2 = b_{k, k+1} &= -\frac{a_{k+1,k}b_{k,k}}{a_{k+1,k+1}}\,,
\:\:\text{1 multiplication, 1 division;}
 \\
\cdots \\
(\mathbf{b}_k^l)_{n-k+1} = b_{k, n} &=
-\frac{1}{a_{n,n}}\sum_{j=n-k+1}^n
a_{j,n}b_{k,j}\,,
\:\:\text{2k-1 arithmetic operations.}
\end{align*}
This gives $j$ multiplications/divisions and $j-1$ additions/subtractions for the $j$'th element of the $k$'th column of $B$, so upon adding all operations for all $k$ rows, we have that the total number of arithmetic operations is
\newcommand{\dk}{\:\mbox{d}k}
\begin{align*}
&\sum_{k=1}^n
\sum_{j= 1}^{n-k+1}
(2j - 1)
=
\sum_{k=1}^n
(n-k+1)(n-k+2) - (n-k+1)
=
\sum_{k=1}^n
(n-k+1)^2 \\
&\approx
\int_1^n (n-k+1)^2 \dk
=
-\frac{1}{3}[(n-k+1)^3]_{k=1}^n = -\frac{1}{9} 
+ \frac{n^3}{3} \approx \frac{n^3}{3}\,.
\end{align*}
Hence, the algorithm is of order $\mathcal{O}(n^3)$.
\section*{Problem 3}
The code snippets compute the inverse of an upper diagonal matrix; it does this via a row-oriented back-solve very similar to Algorithm $2.7$. Finally, the program prints the product $A U$, where $U$ is the computed inverse matrix of $A$, hopefully giving the identity matrix as a result. The variables $r$ and $k$ denote the rows and columns of $U$ respectively.
\end{document}
