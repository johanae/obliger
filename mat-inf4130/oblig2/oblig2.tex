\documentclass[11pt]{amsart}
%\usepackage{pstricks,pst-plot}
\usepackage{tikz}
\usepackage[utf8]{inputenc}
\usepackage{listings}

\definecolor{mygreen}{rgb}{0,0.6,0}
\definecolor{mygray}{rgb}{0.5,0.5,0.5}
\definecolor{mymauve}{rgb}{0.58,0,0.82}

\lstset{ %
  backgroundcolor=\color{white},   % choose the background color; you must add \usepackage{color} or \usepackage{xcolor}
  basicstyle=\footnotesize,        % the size of the fonts that are used for the code
  breakatwhitespace=false,         % sets if automatic breaks should only happen at whitespace
  breaklines=true,                 % sets automatic line breaking
  captionpos=b,                    % sets the caption-position to bottom
  commentstyle=\color{mygreen},    % comment style
  deletekeywords={...},            % if you want to delete keywords from the given language
  escapeinside={\%*}{*)},          % if you want to add LaTeX within your code
  extendedchars=true,              % lets you use non-ASCII characters; for 8-bits encodings only, does not work with UTF-8
  frame=single,                    % adds a frame around the code
  keepspaces=true,                 % keeps spaces in text, useful for keeping indentation of code (possibly needs columns=flexible)
  keywordstyle=\color{blue},       % keyword style
  language=Octave,                 % the language of the code
  morekeywords={*,...},            % if you want to add more keywords to the set
  numbers=left,                    % where to put the line-numbers; possible values are (none, left, right)
  numbersep=5pt,                   % how far the line-numbers are from the code
  numberstyle=\tiny\color{mygray}, % the style that is used for the line-numbers
  rulecolor=\color{black},         % if not set, the frame-color may be changed on line-breaks within not-black text (e.g. comments (green here))
  showspaces=false,                % show spaces everywhere adding particular underscores; it overrides 'showstringspaces'
  showstringspaces=false,          % underline spaces within strings only
  showtabs=false,                  % show tabs within strings adding particular underscores
  stepnumber=2,                    % the step between two line-numbers. If it's 1, each line will be numbered
  stringstyle=\color{mymauve},     % string literal style
  tabsize=2,                       % sets default tabsize to 2 spaces
  title=\lstname                   % show the filename of files included with \lstinputlisting; also try caption instead of title
}

\newtheorem{theorem}{Theorem}[section]
\newtheorem{lemma}[theorem]{Lemma}
\newtheorem{corollary}[theorem]{Corollary}
\newtheorem{prop}[theorem]{Proposition}
\newtheorem{non}[theorem]{Nonsense}
\newtheorem{observation}[theorem]{Observation}

\theoremstyle{definition}
\newtheorem{definition}[theorem]{Definition}
\newtheorem{example}[theorem]{Example}
\newtheorem{xca}[theorem]{Exercise}
\newtheorem{problem}[theorem]{Problem}
\newtheorem{remark}[theorem]{Remark}
\newtheorem{nremark}[theorem]{Bemerkning}
\newtheorem{question}[theorem]{Question}
\newtheorem{conjecture}[theorem]{Conjecture}
\newtheorem{improvement}[theorem]{Improvement}
\newtheorem{discus}[theorem]{Discus}
\newtheorem{ptheorem}[theorem]{Possible Theorem}
\newtheorem{project}[theorem]{Project}

\newcommand\hra{\hookrightarrow}

\newcommand{\cA}{\mathcal{A}}
\newcommand{\cB}{\mathcal{B}}
\newcommand{\cR}{\mathcal{R}}
\newcommand{\cC}{\mathcal{C}}
\newcommand{\cD}{\mathcal{D}}
\newcommand{\cE}{\mathcal{E}}
\newcommand{\cF}{\mathcal{F}}
\newcommand{\cG}{\mathcal{G}}
\newcommand{\cH}{\mathcal{H}}
\newcommand{\cI}{\mathcal{I}}
\newcommand{\cJ}{\mathcal{J}}
\newcommand{\cK}{\mathcal{K}}
\newcommand{\cL}{\mathcal{L}}
\newcommand{\cN}{\mathcal{N}}
\newcommand{\cO}{\mathcal{O}}
\newcommand{\cS}{\mathcal{S}}
\newcommand{\cZ}{\mathcal{Z}}
\newcommand{\cP}{\mathcal{P}}
\newcommand{\cT}{\mathcal{T}}
\newcommand{\cU}{\mathcal{U}}
\newcommand{\cV}{\mathcal{V}}
\newcommand{\cX}{\mathcal{X}}
\newcommand{\cW}{\mathcal{W}}


\newcommand{\A}{\mathbb{A}}
\newcommand{\B}{\mathbb{B}}
\newcommand{\C}{\mathbb{C}}
\newcommand{\D}{\mathbb{D}}
\renewcommand{\H}{\mathbb{H}}
\newcommand{\N}{\mathbb{N}}
\newcommand{\Q}{\mathbb{Q}}
\newcommand{\Z}{\mathbb{Z}}
\renewcommand{\P}{\mathbb{P}}
\newcommand{\R}{\mathbb{R}}
\newcommand{\U}{\mathbb{U}}
\newcommand{\bT}{\mathbb{T}}
\newcommand{\bD}{\mathbb{D}}

\newcommand\Subset{\subset\subset}
\newcommand\wt{\widetilde}
\newcommand\ddt{\frac{d}{dt}}


\newcommand{\Aut}{\mathop{{\rm Aut}}}
\newcommand{\seq}[1]{\{#1\}_{n\in\N}}

\def\di{\partial}
\def\bs{\backslash}
\def\e{\epsilon}


\setlength\parindent{0pt} % Removes all indentation from paragraphs

\numberwithin{equation}{section}


%
%
%  THE DOCUMENT
%
%

\begin{document}
\title[MAT-INF4130]{Compulsory exercise 1}

\author{Johan Åmdal Eliassen}

\maketitle
\subjclass[Mat-inf4130 2015]{}

\date{\today}
\keywords{}

\section{Givens rotations to solve $A\mathbf{x}= \mathbf{b}$}
General idea, using roughly the same notation as in the book:
\begin{equation*}\left(
\begin{matrix}
x&x&x&x \\
x&x&x&x \\
x&x&x&x \\
x&x&x&x \\
\end{matrix}
\right.
\left|
\begin{matrix}
b_1 \\
b_2 \\
b_3 \\
b_4
\end{matrix}
\right)
\rightarrow
\left(
\begin{matrix}
\mathbf{r}&x^*&x^*&x^* \\
x&x&x&x \\
x&x&x&x \\
\mathbf{0}&x^*&x^*&x^* \\
\end{matrix}
\right.
\left|
\begin{matrix}
b_1 \\
b_2 \\
b_3^* \\
b_4^*
\end{matrix}
\right)
\rightarrow
\left(
\begin{matrix}
\mathbf{r}^*&x^{**}&x^{**}&x^{**}\\
x&x&x&x \\
\mathbf{0}&x^*&x^*&x^* \\
0&x^*&x^*&x^* \\
\end{matrix}
\right.
\left|
\begin{matrix}
b_1 \\
b_2^{**} \\
b_3^* \\
b_4^*
\end{matrix}
\right)
\end{equation*}
\begin{equation*}
\rightarrow
\left(
\begin{matrix}
r_{1,1}&r_{1,2}&r_{1,3}&r_{1,4} \\
\mathbf{0}&x^{**}&x^{**}&x^{**} \\
0&x^{**}&x^{**}&x^{**}\\
0&x^*&x^*&x^* \\
\end{matrix}
\right.
\left|
\begin{matrix}
c_1 \\
b_2^{**} \\
b_3^{**} \\
b_4^*
\end{matrix}
\right)
\rightarrow \cdots
\end{equation*}
with the asterix superscripts denoting the number of times 
that element has been changed via matrix multiplication.

In Python, this becomes:
\begin{lstlisting}
def rotsolve(A, b):
    """
    Solve the system Ax=b where A any N x N matrix, again using Givens rotations. Modified from rothesstri(A, b).
    
    A: A matrix. Nonsingular ones only.
    b: The right hand side.
    """
    n = shape(A)[0]
    A = hstack([A, b])
    for k in xrange(n-1):   # columns
        for j in xrange(n - 2, k -1, -1):   # rows
            r = linalg.norm([ A[j , k] , A[j + 1, k] ])
            if r>0:
                c=A[j, k]/r; s=A[j + 1, k]/r
                A[[j, j + 1],(k + 1):(n + 1)] = \
                    [[c, s],[-s, c]]*A[[j, j + 1],(k + 1):(n + 1)]
            A[j, k] = r; A[j+1,k] = 0
    z = A[:, n].copy()
    rbacksolve(A[:, :n], z, n)
    return z
\end{lstlisting}
The program itself is enclosed, obviously.

\subsection*{An aside of arguable interest}Comparing errors (problem 3) with a friend gave very different results; there are obviously many ways to implement this solver. The following is a perhaps more intuitive approach:
\begin{equation*}\left(
\begin{matrix}
x&x&x&x \\
x&x&x&x \\
x&x&x&x \\
x&x&x&x \\
\end{matrix}
\right.
\left|
\begin{matrix}
b_1 \\
b_2 \\
b_3 \\
b_4
\end{matrix}
\right)
\rightarrow
\left(
\begin{matrix}
\mathbf{r}&x^*&x^*&x^* \\
x&x&x&x \\
x&x&x&x \\
\mathbf{0}&x^*&x^*&x^* \\
\end{matrix}
\right.
\left|
\begin{matrix}
b_1^* \\
b_2 \\
b_3 \\
b_4^*
\end{matrix}
\right)
\rightarrow
\left(
\begin{matrix}
\mathbf{r}&x^*&x^*&x^* \\
x&x&x&x \\
\mathbf{0}&x^{*}&x^{*}&x^{*}\\
0&x^*&x^*&x^* \\
\end{matrix}
\right.
\left|
\begin{matrix}
b_1^{**} \\
b_2 \\
b_3^* \\
b_4^*
\end{matrix}
\right)
\end{equation*}
\begin{equation*}
\rightarrow
\left(
\begin{matrix}
r_{1,1}&r_{1,2}&r_{1,3}&r_{1,4} \\
\mathbf{0}&x^*&x^*&x^* \\
0&x^*&x^*&x^*\\
0&x^*&x^*&x^* \\
\end{matrix}
\right.
\left|
\begin{matrix}
c_1 \\
b_2^{**} \\
b_3^{**} \\
b_4^*
\end{matrix}
\right)
\rightarrow \cdots\\.
\end{equation*}
with code:
\begin{lstlisting}
def rotsolveAlt(A, b):
    """
    Solve the system Ax=b where A any N x N matrix, again using Givens
    rotations. Modified from rothesstri(A, b).
    
    A: A matrix. Nonsingular ones preferred, obviously.
    b: The right hand side.
    """
    n = shape(A)[0]
    A = hstack([A, b])
    for k in xrange(n-1):   # columns
        for j in xrange(k + 1, n):   # rows
            r = linalg.norm([ A[k , k] , A[j, k] ], 2)
            if r>0:
                c=A[k, k]/r; s=A[j, k]/r
                A[[k, j],(k + 1):(n+1)] = \
                    mat([[c, s],[-s, c]])* \
                    A[[k, j],(k + 1):(n+1)]
            A[k, k] = r; A[j,k] = 0
    z = A[:, n].copy()
    rbacksolve(A[:, :n], z, n)
    return z
\end{lstlisting}
\section{Complexity of the algorithm}
Taking, for simplicity, the computation of $r$ to be four operations (squaring, squaring, adding, taking the square root), and having that computing $c$ and $s$ each takes one operation, we  have besides this each inner step in the algorithm (i.e. for each $j$) containing a multiplication of a $(2\times 2)$ matrix with a $(2 \times (n - k + 1))$ matrix, for a total of $6(n-k + 1) + 6 = 6(n - k + 2)$ operations. 

There are $n-k$ $j$-steps, making for a total of $6(n-k + 2)(n-k)$ arithmetic operations per $k$-step. Thus, the total number of operations are
\newcommand{\dk}{\:\mbox{d}k}
\newcommand{\dl}{\:\mbox{d}l}
\begin{equation}
\sum_{k = 1}^{n-1}
6(n - k + 2)(n-k)
=
6
\sum_{l = 1}^{n-1}l(l+2)
\end{equation}
\begin{equation}
\approx 6\int_0^{n-1} l^2 + 2l \dl = 
2(n-1)^3 + 2(n-1) = 2n^3 - 6n^2 + 8n - 1 \approx \underline{2n^3\,.}
\end{equation}
as $n$ becomes large.

Finally, we end up with a system $U\mathbf{x} = \mathbf{c}$, where $U$ is upper triangular, at which point the backsolver can be applied, but that algorithm has takes $n^2$ operations, thus not changing the result.
$\qed$
\section{Also}
The code is enclosed, but:
\begin{lstlisting}
if __name__ == '__main__':
    
    N = 20
    
    H_ = hilbert(N)
    xe_ = mat(ones(N)).T
    err = empty(N)
    errAlt = empty(N)
    iterations = arange(N)
    for n in iterations:
#        H = H_[:n+1, :n+1]; xe = xe_[:n+1]
        H = hilbert(n+1); xe = mat(ones(n+1)).T
        b = H*xe
        x = rotsolve(H, b)
        err[n] = linalg.norm(x - xe, 2)
        y = rotsolveAlt(H, b)
        errAlt[n] = linalg.norm(y - xe, 2)
    
    plt.plot(1+iterations, err, 'b')
    plt.plot(1 +iterations, errAlt, 'r')
    plt.legend(['Algorithm 1','Algorithm 2'])
    plt.xlabel('size of matrix [n]')
    plt.ylabel('||x - x_e||_2')
    plt.show()
\end{lstlisting}
As noted, there were two implementations of \verb|rotsolve|; I have plotted the error of both.
\includegraphics[scale=•]{•}=.5]{problem3}
\end{document}