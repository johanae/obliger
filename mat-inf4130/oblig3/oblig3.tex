\documentclass[11pt]{amsart}
%\usepackage{pstricks,pst-plot}
\usepackage{cancel}
\usepackage{tikz}
\usepackage[utf8]{inputenc}
\usepackage{listings}

\definecolor{mygreen}{rgb}{0,0.6,0}
\definecolor{mygray}{rgb}{0.5,0.5,0.5}
\definecolor{mymauve}{rgb}{0.58,0,0.82}

\lstset{ %
  backgroundcolor=\color{white},   % choose the background color; you must add \usepackage{color} or \usepackage{xcolor}
  basicstyle=\footnotesize,        % the size of the fonts that are used for the code
  breakatwhitespace=false,         % sets if automatic breaks should only happen at whitespace
  breaklines=true,                 % sets automatic line breaking
  captionpos=b,                    % sets the caption-position to bottom
  commentstyle=\color{mygreen},    % comment style
  deletekeywords={...},            % if you want to delete keywords from the given language
  escapeinside={\%*}{*)},          % if you want to add LaTeX within your code
  extendedchars=true,              % lets you use non-ASCII characters; for 8-bits encodings only, does not work with UTF-8
  frame=single,                    % adds a frame around the code
  keepspaces=true,                 % keeps spaces in text, useful for keeping indentation of code (possibly needs columns=flexible)
  keywordstyle=\color{blue},       % keyword style
  language=Octave,                 % the language of the code
  morekeywords={*,...},            % if you want to add more keywords to the set
  numbers=left,                    % where to put the line-numbers; possible values are (none, left, right)
  numbersep=5pt,                   % how far the line-numbers are from the code
  numberstyle=\tiny\color{mygray}, % the style that is used for the line-numbers
  rulecolor=\color{black},         % if not set, the frame-color may be changed on line-breaks within not-black text (e.g. comments (green here))
  showspaces=false,                % show spaces everywhere adding particular underscores; it overrides 'showstringspaces'
  showstringspaces=false,          % underline spaces within strings only
  showtabs=false,                  % show tabs within strings adding particular underscores
  stepnumber=2,                    % the step between two line-numbers. If it's 1, each line will be numbered
  stringstyle=\color{mymauve},     % string literal style
  tabsize=2,                       % sets default tabsize to 2 spaces
  title=\lstname                   % show the filename of files included with \lstinputlisting; also try caption instead of title
}

\newtheorem{theorem}{Theorem}[section]
\newtheorem{lemma}[theorem]{Lemma}
\newtheorem{corollary}[theorem]{Corollary}
\newtheorem{prop}[theorem]{Proposition}
\newtheorem{non}[theorem]{Nonsense}
\newtheorem{observation}[theorem]{Observation}

\theoremstyle{definition}
\newtheorem{definition}[theorem]{Definition}
\newtheorem{example}[theorem]{Example}
\newtheorem{xca}[theorem]{Exercise}
\newtheorem{problem}[theorem]{Problem}
\newtheorem{remark}[theorem]{Remark}
\newtheorem{nremark}[theorem]{Bemerkning}
\newtheorem{question}[theorem]{Question}
\newtheorem{conjecture}[theorem]{Conjecture}
\newtheorem{improvement}[theorem]{Improvement}
\newtheorem{discus}[theorem]{Discus}
\newtheorem{ptheorem}[theorem]{Possible Theorem}
\newtheorem{project}[theorem]{Project}

\newcommand\hra{\hookrightarrow}

\newcommand{\cA}{\mathcal{A}}
\newcommand{\cB}{\mathcal{B}}
\newcommand{\cR}{\mathcal{R}}
\newcommand{\cC}{\mathcal{C}}
\newcommand{\cD}{\mathcal{D}}
\newcommand{\cE}{\mathcal{E}}
\newcommand{\cF}{\mathcal{F}}
\newcommand{\cG}{\mathcal{G}}
\newcommand{\cH}{\mathcal{H}}
\newcommand{\cI}{\mathcal{I}}
\newcommand{\cJ}{\mathcal{J}}
\newcommand{\cK}{\mathcal{K}}
\newcommand{\cL}{\mathcal{L}}
\newcommand{\cN}{\mathcal{N}}
\newcommand{\cO}{\mathcal{O}}
\newcommand{\cS}{\mathcal{S}}
\newcommand{\cZ}{\mathcal{Z}}
\newcommand{\cP}{\mathcal{P}}
\newcommand{\cT}{\mathcal{T}}
\newcommand{\cU}{\mathcal{U}}
\newcommand{\cV}{\mathcal{V}}
\newcommand{\cX}{\mathcal{X}}
\newcommand{\cW}{\mathcal{W}}


\newcommand{\A}{\mathbb{A}}
\newcommand{\B}{\mathbb{B}}
\newcommand{\C}{\mathbb{C}}
\newcommand{\D}{\mathbb{D}}
\renewcommand{\H}{\mathbb{H}}
\newcommand{\N}{\mathbb{N}}
\newcommand{\Q}{\mathbb{Q}}
\newcommand{\Z}{\mathbb{Z}}
\renewcommand{\P}{\mathbb{P}}
\newcommand{\R}{\mathbb{R}}
\newcommand{\U}{\mathbb{U}}
\newcommand{\bT}{\mathbb{T}}
\newcommand{\bD}{\mathbb{D}}

\newcommand\Subset{\subset\subset}
\newcommand\wt{\widetilde}
\newcommand\ddt{\frac{d}{dt}}


\newcommand{\Aut}{\mathop{{\rm Aut}}}
\newcommand{\seq}[1]{\{#1\}_{n\in\N}}

\def\di{\partial}
\def\bs{\backslash}
\def\e{\epsilon}


\setlength\parindent{0pt} % Removes all indentation from paragraphs

\numberwithin{equation}{section}


%
%
%  THE DOCUMENT
%
%

\begin{document}
\title{Compulsory exercise 3}
%\subjclass[Mat-inf4130 2015]{}
\author{Johan Åmdal Eliassen}

\maketitle

\date{\today}
%\keywords{}

\section{Exercise 8.10: Fitting a circle to points}
Given points $\{t_i, y_i\}_{i=1}^m \subset \R^2$, $m \geq 3$, we wish to fit a circle to these points. That is, we wish to satisfy
\begin{equation}\tag{*}\label{circle}
(t_i - c_1)^2 + (y_i - c_2)^2 = r^2\,,\:\:1 \leq i \leq m\,,
\end{equation}
hence $c_1$, $c_2$ and $r$ are the unknowns to be solved for.
\subsection*{a}
We have \begin{equation}
\eqref{circle} = t_i^2 + y_i^2 + c_1^2 + c_2^2 - 2c_1t_i - 2c_2 y_i\,,
\end{equation}
or equivalently,
\begin{equation}
2c_1 t_i^2 + 2c_2 y_i^2 - c_1^2 - c_2^2 + r^2
= t_i^2 + y_i^2\,.
\end{equation}
Set $x_1 = 2c_1$, $x_2 = 2c_2$, and $x_3 = -c_1^2 - c_2^2 + r^2$, and this becomes
\begin{equation}
t_i x_1 + y_i x_2 + x_3 = t_i^2 + y_i^2\,,
\end{equation}
or in abbreviated matrix form,
\begin{equation}
\underbrace{
\left(
\begin{matrix}
t_i, & y_i, & 1
\end{matrix}
\right)}_{=: \mathbf{A}}
\left(
\begin{matrix}
x_1, & x_2, & x_3
\end{matrix}
\right)^T
=
\underbrace{
(t_i^2 + y_i^2)}_{=: \mathbf{b}}\,,\:\:1\leq i \leq m\,.
\end{equation}
If we can solve the linear system above, it is trivial to derive the original unknowns. We have $c_1 = x_1/2$, $c_2 = x_2/2$ and $r^2 = c_1^2 + c_2^2 + x_3$.

\subsection*{b}
LSQ: minimize $E$ given by
\begin{align*}
E(\mathbf{x}
&=
||A\mathbf{x}-  b||_2^2
=
\sum_{i=1}^m\left(
(t_i x_1 + y_i x_2 + x_3) - (t_i^2 + y_i^2)\right)^2 \\
&=
\sum_{i=1}^m
\left(
t_i(x_1 - t_i) + y_i(x_2 - y_i) + x_3
\right)^2\,.
\end{align*}
While it's by far the most unwieldy way to solve this problem, one way to proceed is to set the gradient of $E$ to zero.
\begin{equation}
\frac{\partial E}{\partial x_j}
=
\sum_{i=1}^m
2q_{i,j} (t_i(x_1 - t_i) + y_i(x_2 - y_i) + x_3) = 0\,,
\end{equation}
with $q_{i,j} = t_i, y_i$ or $1$ for $j = 1,2,3$ respectively.
Rewritten, we obtain the three equations
\begin{align}
&
x_1
\sum_{i=1}^m t_i^2 + x_2 \sum_{i=1}^m t_i y_i + x_3 \sum_{i=1}^m t_i
 = \sum_{i=1}^m(t_i^3 + y_i^2 t_i) \,,\\
&x_1 \sum_{i=1}^m t_i y_i + x_2 \sum_{i=1}^m y_i^2 + x_3 \sum_{i=1}^m y_i = \sum_{i=1}^m\left(
t_i**2 y_i + y_i^3
\right) \,,\\
&x_1 \sum_{i=1}^m t_i + x_2 \sum_{i=1}^m y_i + m x_3 = \sum_{i=1}^m \left(
t_i^2 + y_i^2
\right)\,.
\end{align}
In matrix form, $\mathbf{B}\mathbf{x} = \mathbf{c}$.
\subsection*{c}
Two more or less obvious conditions are sufficient to ensure that $\mathbf{A}$ has full column rank: first, there must be at least three distinct points $(t_i, y_i)$, and secondly, these three points must not all lie on the same line $y(t) = \alpha t$, for some $\alpha \in \R$. If these conditions are met, $\mathbf{A}$ is guaranteed to have three linearly independed rows, i.e. a row rank of 3, which implies full column rank. These conditions are necessary even if the system is otherwise overdetermined.
\subsection*{d}
Let $\{(t_i, y_i)\}_{i=1}^m = \{(1, 4),(3, 2),(1, 0)\}$, for $m=3$. Three points on a circle uniquely determines it, so we may simply plug the points into $\mathbf{A}$ to obtain
$\mathbf{x} = (2,4,-1)^T$, or
$c_1 = 1$, $c_2 = 2$, $r = 2$.

To verify that the calculations in (b) were correct, inserting the numbers into $\mathbf{B}$ and $\mathbf{c}$ as obtained there, I get
\begin{equation}
\left(
\begin{matrix}
11&10&5\\
10&20&6\\
5&6&3
\end{matrix}
\right)
\mathbf{x}
=
\left(
\begin{matrix}
57\\94\\31
\end{matrix}
\right)\,,
\end{equation}
which yields the same results.

\section*{Exercise 8.24}
\renewcommand{\A}{\mathbf{A}}
\newcommand{\Ast}{\mathbf{A^*}}
\newcommand{\Gaist}{(\mathbf{A}^\dagger)^*}
\newcommand{\Gai}{\mathbf{A^\dagger}}

\renewcommand{\B}{\mathbf{B}}
\newcommand{\Bst}{\mathbf{B}^*}
\newcommand{\Binv}{\mathbf{C}}
\newcommand{\Bstinv}{(\mathbf{C^*}}
Given $\A \in \C^{m\times n}$, from previous exercises (8.17, 8.18), we have that $\Gai$ is the unique matrix satisfying the following:
\begin{enumerate}
\item $(\A\Gai)^* = \A\Gai$,
\item $(\Gai \A)^* = \Gai \A$,
\item $\Gai \A \Gai = \Gai$,
\item $\A \Gai \A = \A$.
\end{enumerate}
Hence, for a matrix $\B$, to show that a matrix $\Binv$ is the generalized inverse of $\B$ is to show that $\B$ and $\Binv$ together satisfy the four identities above.

\subsection*{a} Show that $(\mathbf{A}^*)^\dagger = (\Gai)^*$:
For notational simplicity, let $\mathbf{B} = \A^*$, $\mathbf{C} = \Gaist$.

(1) and (2) trivially hold; for (3) we have
\begin{align*}
\Binv \B \Binv
=((\Binv \B \Binv)^*)^*
= (\Binv^* \B^* \Binv^*)^*
=(\Gai \A \Gai)^* = \Gaist = \Binv\,,
\end{align*}
and likewise for (4),
\begin{equation}
\B \Binv \B
=((\B \Binv \B)^*)^*
=(\A \Gai \A)^* = \A^* = \B\,.
\end{equation}
Hence $\Binv = \Gaist$ is indeed the generalized inverse of $\B = \Ast$.
\qed
\subsection*{b}
Show that $(\Gai)^\dagger = \A$.

(3) and (4) follow from (4) and (3) for $\A$.
For (1) we have:
\begin{equation*}
(\Gai \A)^*
= \Ast \Gaist\,.
\end{equation*}
Now, from (a) this becomes
\begin{equation*}
(...) = \Ast (\Ast)^\dagger
= (\Ast (\Ast)^\dagger)^*
= \Gai \A\,.
\end{equation*}
again using (a).

The procedure for showing (2) is practically identical:
\begin{equation*}
(\A \Gai)^*
= \Gaist \Ast\
= (\Ast)^\dagger \Ast
= ((\Ast)^\dagger \Ast )^*
= \A\ \Gai,.
\end{equation*}
\qed
\subsection*{c}
Show: $(\alpha \A)^\dagger = \frac{1}{\alpha} \A^\dagger$, $\alpha \in \C\setminus \{0\}$.

This is more or less trivial. For (1), we have
$(\alpha \A \frac{1}{\alpha}\Gai)^* = (\A \Gai)^* = \A \Gai = \alpha \A \frac{1}{\alpha} \Gai$, and similarly for (2), and for (3) and (4),
\begin{equation}
\frac{1}{\alpha}\Gai \alpha \A \frac{1}{\alpha}\Gai
=
\frac{1}{\alpha}(\Gai \A \Gai)
=
\frac{1}{\alpha} \Gai\,,
\end{equation}
and
\begin{equation}
\alpha \A \frac{1}{\alpha}\Gai \alpha \A
=
\alpha \A \Gai \A = \frac{1}{\alpha} \A\,.
\end{equation}
\qed
\end{document}