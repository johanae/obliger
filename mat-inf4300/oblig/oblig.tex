\documentclass[11pt]{amsart}
%\usepackage{pstricks,pst-plot}
\usepackage{cancel}
\usepackage{tikz}
\usepackage[utf8]{inputenc}
\usepackage{listings}

\definecolor{mygreen}{rgb}{0,0.6,0}
\definecolor{mygray}{rgb}{0.5,0.5,0.5}
\definecolor{mymauve}{rgb}{0.58,0,0.82}

\lstset{ %
  backgroundcolor=\color{white},   % choose the background color; you must add \usepackage{color} or \usepackage{xcolor}
  basicstyle=\footnotesize,        % the size of the fonts that are used for the code
  breakatwhitespace=false,         % sets if automatic breaks should only happen at whitespace
  breaklines=true,                 % sets automatic line breaking
  captionpos=b,                    % sets the caption-position to bottom
  commentstyle=\color{mygreen},    % comment style
  deletekeywords={...},            % if you want to delete keywords from the given language
  escapeinside={\%*}{*)},          % if you want to add LaTeX within your code
  extendedchars=true,              % lets you use non-ASCII characters; for 8-bits encodings only, does not work with UTF-8
  frame=single,                    % adds a frame around the code
  keepspaces=true,                 % keeps spaces in text, useful for keeping indentation of code (possibly needs columns=flexible)
  keywordstyle=\color{blue},       % keyword style
  language=Octave,                 % the language of the code
  morekeywords={*,...},            % if you want to add more keywords to the set
  numbers=left,                    % where to put the line-numbers; possible values are (none, left, right)
  numbersep=5pt,                   % how far the line-numbers are from the code
  numberstyle=\tiny\color{mygray}, % the style that is used for the line-numbers
  rulecolor=\color{black},         % if not set, the frame-color may be changed on line-breaks within not-black text (e.g. comments (green here))
  showspaces=false,                % show spaces everywhere adding particular underscores; it overrides 'showstringspaces'
  showstringspaces=false,          % underline spaces within strings only
  showtabs=false,                  % show tabs within strings adding particular underscores
  stepnumber=2,                    % the step between two line-numbers. If it's 1, each line will be numbered
  stringstyle=\color{mymauve},     % string literal style
  tabsize=2,                       % sets default tabsize to 2 spaces
  title=\lstname                   % show the filename of files included with \lstinputlisting; also try caption instead of title
}

\newtheorem{theorem}{Theorem}[section]
\newtheorem{lemma}[theorem]{Lemma}
\newtheorem{corollary}[theorem]{Corollary}
\newtheorem{prop}[theorem]{Proposition}
\newtheorem{non}[theorem]{Nonsense}
\newtheorem{observation}[theorem]{Observation}

\theoremstyle{definition}
\newtheorem{definition}[theorem]{Definition}
\newtheorem{example}[theorem]{Example}
\newtheorem{xca}[theorem]{Exercise}
\newtheorem{problem}[theorem]{Problem}
\newtheorem{remark}[theorem]{Remark}
\newtheorem{nremark}[theorem]{Bemerkning}
\newtheorem{question}[theorem]{Question}
\newtheorem{conjecture}[theorem]{Conjecture}
\newtheorem{improvement}[theorem]{Improvement}
\newtheorem{discus}[theorem]{Discus}
\newtheorem{ptheorem}[theorem]{Possible Theorem}
\newtheorem{project}[theorem]{Project}

\newcommand\hra{\hookrightarrow}

\newcommand{\cA}{\mathcal{A}}
\newcommand{\cB}{\mathcal{B}}
\newcommand{\cR}{\mathcal{R}}
\newcommand{\cC}{\mathcal{C}}
\newcommand{\cD}{\mathcal{D}}
\newcommand{\cE}{\mathcal{E}}
\newcommand{\cF}{\mathcal{F}}
\newcommand{\cG}{\mathcal{G}}
\newcommand{\cH}{\mathcal{H}}
\newcommand{\cI}{\mathcal{I}}
\newcommand{\cJ}{\mathcal{J}}
\newcommand{\cK}{\mathcal{K}}
\newcommand{\cL}{\mathcal{L}}
\newcommand{\cN}{\mathcal{N}}
\newcommand{\cO}{\mathcal{O}}
\newcommand{\cS}{\mathcal{S}}
\newcommand{\cZ}{\mathcal{Z}}
\newcommand{\cP}{\mathcal{P}}
\newcommand{\cT}{\mathcal{T}}
\newcommand{\cU}{\mathcal{U}}
\newcommand{\cV}{\mathcal{V}}
\newcommand{\cX}{\mathcal{X}}
\newcommand{\cW}{\mathcal{W}}

\newcommand{\dy}{\mbox{d}y}
\newcommand{\dt}{\mbox{d}t}
\newcommand{\ds}{\mbox{d}s}


\newcommand{\A}{\mathbb{A}}
\newcommand{\B}{\mathbb{B}}
\newcommand{\C}{\mathbb{C}}
\newcommand{\D}{\mathbb{D}}
\renewcommand{\H}{\mathbb{H}}
\newcommand{\N}{\mathbb{N}}
\newcommand{\Q}{\mathbb{Q}}
\newcommand{\Z}{\mathbb{Z}}
\renewcommand{\P}{\mathbb{P}}
\newcommand{\R}{\mathbb{R}}
\newcommand{\U}{\mathbb{U}}
\newcommand{\bT}{\mathbb{T}}
\newcommand{\bD}{\mathbb{D}}

\newcommand\Subset{\subset\subset}
\newcommand\wt{\widetilde}
\newcommand\ddt{\frac{d}{dt}}


\newcommand{\Aut}{\mathop{{\rm Aut}}}
\newcommand{\seq}[1]{\{#1\}_{n\in\N}}

\def\di{\partial}
\def\bs{\backslash}
\def\e{\epsilon}


\setlength\parindent{0pt} % Removes all indentation from paragraphs

\numberwithin{equation}{section}


%
%
%  THE DOCUMENT
%
%

\begin{document}
\title{Mandatory assignment - MAT-INF4300}
%\subjclass[Mat-inf4130 2015]{}
\author{Johan Åmdal Eliassen}

\maketitle

\date{\today}
%\keywords{}
\section{Problem 1}
\subsection*{a}
Assume $u \in C^3(\R^n \times (0,\infty))$ solves $u_t - \Delta u = 0$.

Let $u_\lambda(x,t) = u(\lambda x, \lambda^2 t)$, $\lambda \in \R$.

\newcommand{\ul}{u_\lambda}
Then $(\ul)_t(x,t) = \lambda^2 u_t(\lambda x, \lambda^2 x)$, and $(\ul)_{x_i, x_i}(x, t) = \lambda^2 u_{x_i,x_i}(\lambda x, \lambda^2 t)$, and so 
\begin{equation}
((\ul)_t - \Delta \ul)(x,t) = \lambda^2 (u_t(\lambda x, \lambda^2 t) - \Delta u(\lambda x, \lambda^2 t)) = 0\,.
\end{equation}
\qed

\subsection*{b} Let $v(x,t) = x \cdot Du(x,t) + 2tu_t(x,t)$.

(I am assuming now that $u$ is three times continuously differentiable, not just two).

Differentiating $(x, t, \lambda) \mapsto u_\lambda(x, t)$ with regards to $\lambda$ gives $x \cdot Du(\lambda x, \lambda^2 t) + 2t\lambda u(\lambda x, \lambda^2 t) =: w(x,t,\lambda)$; we then have $v(x, t) = w(x, t, 1)$. 

Since $(x,t,\lambda) \mapsto u_\lambda(x,t)$ solves the heat equation for all $\lambda \in \R$, so does $w(x,t,\lambda)$ for all $\lambda \in \R$, and in particular, for $\lambda = 1$. Thus $v$ solves the heat equation.

Slightly more written out, continuity of the derivatives of $u_\lambda$ allows us to interchange the order of differentiation, and so,
\begin{equation}
v_t - \Delta v
=
\left.\left(
 \left(\frac{\partial u_\lambda}{\partial \lambda}
\right)_t 
- \Delta \left(
\frac{\partial u_\lambda}{\partial \lambda}
\right)\right)\right|_{\lambda = 1}
= \left.
\frac{\partial}{\partial \lambda}
\left(
(u_\lambda)_t - \Delta u_\lambda
\right)\right|_{\lambda = 1} = 0\,.
\end{equation}
 \qed
 
\subsection*{c}Let $\eta: \R \to \R$ be convex and twice continuously differentiable, let $u$ solve the heat equation, and set $v(x,t) := \eta(u)$. Then
\begin{align}
v_t &= \frac{\partial u}{\partial t}\eta'(u) = u_t \eta'(u)\,,\\
v_{x_i} &= \frac{\partial u}{\partial x_i}\eta'(u) = u_{x_i} \eta'(u)\,,\\
v_{x_i x_i} &=
u_{x_i}^2 \eta''(u) + u_{x_i x_i} \eta'(u)\,.
\end{align}
Hence $v_t - \Delta v = \eta'(u)(u_t - \Delta u) - \eta''(u)|Du|^2 = -\eta''(u)|Du|^2 \leq 0$ because $\eta''(u) \geq 0$. \qed

\section{Problem 2}
Problem:
\begin{equation} \tag{*}\label{prob2}
\left\{
\begin{array}{cc}
u_t - \Delta u + cu = f\,, & (x,t) \in \R^n\times(0,\infty) \,, \\
u(x,0) = u_0(x)\,, & x\in\R^n\,,
\end{array}
\right.
\end{equation}
for $c \in \R$, $f \in C_c^{2,1}(\R^n \times [0,\infty))$, $u_0 \in C_c(\R^n)$.

\subsection*{a}
Assume there exists a $v$ satisfying
\begin{equation}\tag{**}\label{prob2mod}
\left\{
\begin{array}{cc}
v_t - \Delta v = e^{ct}f\,, & (x,t) \in \R^n\times(0,\infty) \,, \\
v(x,0) = u_0(x)\,, & x\in\R^n\,,
\end{array}
\right.
\end{equation}
Then for $w := e^{-ct}v$ we have
\begin{align} \label{c2}
w_t &= e^{-ct}(-cv + v_t)\,, \\
\label{c21}
\Delta w &= e^{-ct}\Delta v\,.
\end{align}
And so
\begin{equation}
w_t - \Delta w + cw
=
e^{-ct}(-cv + v_t - \Delta v + c v)
=e^{-ct}(v_t - \Delta v) = f\,,
\end{equation}
so $w$ satisfies the first half of \eqref{prob2}.

A solution of \eqref{prob2mod} is 
\begin{equation}
v(x,t) = \int_{\R^n}\!\!\!\!
\Phi(x - y,t) u_0(x)
\:\dy 
+ 
\int_0^t\!\! \int_{\R^n}\!\!\!\!
 \Phi(x - y, t - s)e^{cs}f(y, s) \:\ds\,.
\end{equation}
This is valid because $f$ and $u_0$ have compact support.

Hence, an explicit formula for $u$ satisfying \eqref{prob2} is
\begin{equation}
u(x,t) = e^{-ct}\left(
\int_{\R^n}\!\!\!\!
\Phi(x-  y,t) u_0(x)
\:\dy 
+ 
\int_0^t\!\! \int_{\R^n}\!\!\!\!
 \Phi(x - y, t - s)e^{cs}f(y, s) \:\ds\
\right) 
\,.
\end{equation}
It remains to verify:
\begin{enumerate}
\item that $u$ really is in $C^{2,1}(\R^n \times(0,\infty)$,
\item that $u$ really does satisfy $u_t - \Delta u + cu = f$, and
\item that $u(x,t) \to u_0(x_0)$ whenever $(x,t) \to (x_0,0)$.
\end{enumerate}

\subsubsection*{(1)}
Since $v \in C^{2,1}(\R^n \times (0,\infty))$ and $u$ is a product of a smooth function with $v$, it follows that $u$ is also a member of $C^{2,1}(\R^n \times (0,\infty))$.

\subsubsection*{(2)}
Write $u$ as 
\begin{align*}
u(x,t) &= e^{-ct}\left(
\int_{\R^n}\!\!\!\!
\Phi(x-  y,t) u_0(x)
\:\dy 
+ 
\int_0^t\!\! \int_{\R^n}\!\!\!\!
 \Phi(x - y, t - s)e^{cs}f(y, s) \:\ds\
\right) \\
&=: e^{-ct}(I(x,t) + J(x,t))\,.
\end{align*}
Then, from Theorem 1 on pp. 47 in Evans, $I$ satisfies the homogenous heat equation; additionally, $I(x,t) \to u_0(x_0)$ as $x \to x_0$, $t \downarrow 0$ for $x_0 \in \R^n$.

Likewise, from Theorem 2 on pp. 50 in Evans, $J$ satisfies $J_t - \Delta J = e^{ct}f$. Additionally, $J(x,t) \to 0$ as $x \to x_0$, $t \downarrow 0$ for $x_0 \in \R^n$.

I omit any direct verification, as the calculations involved would necessarily just mirror those in the book.

Hence, we confirm
\begin{align*}
u_t &= e^{-ct}(-c(I + J) + (I_t + J_t)) \,,\\
\Delta u &= e^{-ct}(\Delta I + \Delta J) \,,\\
u_t - \Delta u + c u
&=
e^{-ct}(
-c(I+J) + (I_t - \Delta I) + (J_t - \Delta J) + c(I+J)
)
= f\,.
\end{align*}
\qed
\subsubsection*{(3)}
Again writing $u = e^{-ct}(I + J)$, for any $\epsilon > 0$, we may pick $(x,t)$ with $t > 0$ s.t. $|I(x,t)| < \epsilon/3$ and $|J(x,t) - u_0(x)| < \epsilon/3$ and finally so that $1 - e^{-ct} < \epsilon/3/(|u_0(x_0)| + \epsilon/3)$.
\begin{align*}
|u(x,t) - u_0(x)|
&= 
|e^{-ct}I(x,t) + (e^{-ct}J(x,t) - u_0(x_0)| \\
&\leq |e^{-ct}I(x,t) - u_0(x_0)|
+ e^{-ct}|J(x,t)|\,.
\end{align*}
Since $e^{-ct} = 1 - \delta$ for some $\delta > 0$, this can be written
\begin{align*}
(\cdots) &=
|(1-\delta)I(x,t) - u_0(x_0)| + e^{-ct}|J(x,t)| \\
&\leq
|I(x,t) - u_0(x_0)| + \delta|I(x,t)| + |J(x,t)|\,.
\end{align*}
Now by choice of $(x,t)$, $\delta = 1 - e^{-ct} < \epsilon/3/(|u_0(x_0)| + \epsilon/3)$, and $|I(x,t)| < |u_0(x,t)| + \epsilon/3)$.
Thus, finally,
\begin{align*}
&< \epsilon/3
+ \delta
(|u_0(x_0)| + \epsilon/3) 
+ \epsilon/3
< \epsilon/3  + \epsilon/3 + \epsilon/3 = \epsilon\,.
\end{align*} \qed

\subsection*{b}
\textbf{Note}
I failed to get the energy bound that was asked for, but I stand by my computations, and the result I got is more than sufficient.
\\

\newcommand{\dx}{\mbox{d}x}
Assume $f \equiv 0$, and that $u \to 0$ as $x \to \infty$. Now proceeding in the reverse direction, let $v := e^{ct}u$. Then as seen, $v$ satisfies the homogenous heat equation, and so it is smooth; hence $u$ is also smooth, and we can differentiate under the integral.

Define the energy $E(t)$ by
\begin{equation}
E(t) = ||u(\cdot, t)||_{L^2(\R^n)} = \int_{\R^n}\!\!\!\!
u(x,t)^2\:\dx\,.
\end{equation}
Then 
\begin{equation}
E'(t) =
\int_{\R^n}\!\!\!\!
2u u_t \:\dx
=
\int_{\R^n}\!\!\!\!
-2cu^2\:\dx  +
\int_{\R^n}\!\!\!\!  u\Delta t
\:\dx\,,
\end{equation}
whence, integrating the last term by parts, we otbain
\begin{equation}
(...) = 
\int_{\R^n}\!\!\!\!
-2cu^2\:\dx
- \int_{\R^n}\!\!\!\!
|Du|^2\:\dx\
= -2cE(t) - \int_{\R^n}\!\!\!\!
|Du|^2\:\dx\ \leq -2cE(t)\,.
\end{equation}
This implies that 
$
2cE(t) + E'(t) \leq 0\,,
$
and so, by Grönwall's inequality, $E(t) \leq e^{-2ct}E(0) = e^{-2ct}||u_0||_{L^2(\R^n)}$.

Assuming we have two solutions $u_1, u_2$ of \eqref{prob2} satisfying $u \to 0$ as $|x| \to \infty$, we let $w := u_1 - u_2$, $E_w := ||w||_{L^2(\R^n)}$.
Then $w$ satisfies
\begin{equation}
\left\{
\begin{array}{cc}
w(x,t) = 0\,, &(x,t)\in\R^n\times(0,\infty)\,, \\
w(x,0) = 0\,, &x\in\R^n\,.
\end{array}
\right.
\end{equation}

As has been shown, this implies $0 \leq E_w(t) \leq e^{-2ct}E_w(0) = 0$, so $w \equiv 0$; hence $u_1 \equiv u_2$, as was to be shown.

\section{Problem 3}
\begin{equation}\tag{***}\label{problem3}
\left\{
\begin{array}{cc}
u_t - \Delta u = -u^3\,, &(x,t) \in \Omega \times (0,\infty)\,,\\
u(x,) = u_0(x)\,, & x\in\R^n\,, \\
u(x,t) = 0\,, & (x,t) \in \partial \Omega \times (0,\infty)\,.
\end{array}
\right.
\end{equation}
with $\Omega \subset \R^n$ is open and bounded, $u_0$ continuous.

Assume there exists a twice continuously differentiable $u$ satisfying \eqref{problem3}.
Then as before, let
\begin{equation}
E(t) := 
||u(\cdot, t)||_{L^2(\Omega)}
=
\int_\Omega\!\!\!
u^2
\:\dx\,.
\end{equation}
Justifying that I can differentiate under the integral sign is a bit finicky: let $v := u^2$ for simplicity of notation. Then $v_t$ exists on $\Omega$ and is bounded wrt $x$, $||v_t(\cdot, t)||_{L^\infty(\Omega)} < \infty$. Fix $t \in (0,\infty)$, and $\epsilon > 0$ s.t. $t-\epsilon > 0$. Let $M = ||v||_{L^{\infty}(\Omega \times (t-\epsilon, t+\epsilon)} < \infty$. Then the function $g(x) = M$ is summable, and dominates $v$ for all $(x,t) \in\Omega\times(t-\epsilon, t+\epsilon)$.

Take some sequence $\{t_n\}_{n\in\N}$ s.t. $t_n \to t$ and $|t_n - t| < \epsilon$. Then for $n\in\N$, by the mean value theorem,
\begin{equation}
\frac{v(x, t_n) - v(x,t)}{t_n - t} = \frac{\partial v(x, \zeta_n)}{\partial t} \leq M\,,
\end{equation}
for some $\zeta_n \in (t_n, t)$.

Thus the sequence of functions $\{w_n\}_{n\in\N} := \frac{v(x,t_n) - v(x,t)}{t_n - t}$ is dominated by the summable function $g(x)$, and so by the Dominated Convergence Theorem,
\begin{equation}
\lim_{n\to\infty}\int_\Omega w_n(x,t)\dx = \int_\Omega
\lim_{n\to\infty} w_n(x,t)\dx
= \int_\Omega v_t(x,t)\dx\,.
\end{equation}
Since this holds for any such sequence, for any $t\in(0,\infty)$, we are free to differentiate $v = u^2$ under the integral sign.
\\
Phew. Now, let's finally do that.
\begin{equation}
E'(t) = 
\int_\Omega\!\!
2u u_t
\:\dx
= 
\int_\Omega\!\!
2u(\Delta u - u^3)
\:\dx\
=
\int_\Omega\!\!
2u\Delta u 
\:\dx
-
\int_\Omega\!\!
2u^4
\:\dx\
\,.
\end{equation}
Now integrate the first by parts and use that $u \equiv 0$ on $\partial \Omega$ to obtain
\begin{equation}
(\cdots) =
-\int_\Omega\!\!
|Du|^2 + 2u^4
\:\dx\ \leq 0,.
\end{equation}
Hence $E(t)$ is a nonincreasing function, and so $E(t) \leq E(0) = ||u_0||_{L^2(\Omega)}$.\qed

\section{Problem 4}
Let $\Omega \subset \R^n$ be bounded and open. Then the task is to show that the Hölder space $C^{0,\gamma}(\Omega)$ with exponent $\gamma \in [0,1)$ is a Banach space.
\\
\newcommand{\cnorm}[1]{||#1||_{C^(\overline{\Omega})}}
\newcommand{\hnorm}[1]{||#1||_{C^{0,\gamma}(\overline{\Omega})}}
\newcommand{\hsnorm}[1]{[#1]_{C^{0,\gamma}(\overline{\Omega})}}

The space is equipped with the norm $\hnorm{\cdot}$ given by
\begin{equation}
\hnorm{\cdot} = \cnorm{\cdot} + \hsnorm{\cdot}\,.
\end{equation}

First, I verify that the $\gamma$'th Hölder seminorm is indeed a seminorm--then it follows that the $\gamma'th$ Hölder norm is a norm.

The two properties

$\hsnorm{u} \geq 0$ 
and
$\hsnorm{\lambda u} = |\lambda| \hsnorm{u}$ are trivial.

For the triangle inequality, we have
\begin{equation}
\hsnorm{u+v}
= 
\sup_{
\substack{
x,y\in\overline{\Omega}\\
x\neq y
}
}
\left\{\left|
\frac{u(x) - u(y)}{|x-y|^\gamma}
+
\frac{v(x) - v(y)}{|x-y|^\gamma}
\right|\right\}\,,
\end{equation}
but since 
\begin{equation}
\left|\frac{u(x) - u(y)}{|x-y|^\gamma}
+
\frac{v(x) - v(y)}{|x-y|^\gamma}\right|
\leq
\frac{|u(x) - u(y)|}{|x-y|^\gamma}
+
\frac{|v(x) - v(y)|}{|x-y|^\gamma}
\end{equation}
for all $x,y \in \overline{\Omega}, x\neq y$, the same goes for its supremum.

Hence $\hsnorm{\cdot}$ is a seminorm as was to be shown.

Now, let $\{u_n\}_{n\in\N} \subset C^{0,\gamma}$ be a Cauchy sequence. Then necessarily it is also a Cauchy sequence in the supremum norm $\cnorm{\cdot}$. Since $(C(\overline{\Omega}), \cnorm{\cdot})$ is complete, there then exists a continuous $u$ s.t. $u_n \to u$ in the supremum norm.

Choose $N \in \N$ s.t. $\cnorm{u - u_n} < \epsilon/2$ and $\hsnorm{u_n - u_m} < \epsilon/2$ for all $n, m\in \N$. Then for all $x,y \in \Omega$, $x \neq y$,
\begin{equation}
\left|\frac{u_m(x) - u_m(y) + u_n(y) - u_n(x)}{|x-y|^\gamma}
\right| < \epsilon/2\,,
\end{equation}
and so because $u_m \to u$ pointwise,
\begin{equation}
\left|
\frac{u(x) - u(y) + u_n(y) - u_n(x)}{|x-y|^\gamma}
\right|
=
\lim_{m\to\infty}
\left|
\frac{
u_m(x) - u_m(y) + u_n(y) - u_n(x)}{|x-y|^\gamma}
\right| < \epsilon/2\,,
\end{equation}
since the above holds for all $m \geq N$. Hence
 $\hsnorm{u - u_n} \leq \epsilon/2$, and so we have $\hnorm{u - u_n} < \epsilon$.
 
It remains to show that $\hnorm{u} < \infty$. But $\cnorm{u} < \infty$, and
\begin{equation}
\hsnorm{u} = \hsnorm{u - u_n + u_n}
\leq \hsnorm{u - u_n} + \hsnorm{u_n} < \infty\,,
\end{equation}
for some $u_n$ satisfying $\hsnorm{u - u_n} < \infty$.

This completes the proof.\qed

\section{Problem 5}
\newcommand{\spt}{\text{spt}}
\newcommand{\wnorm}[2]{||#1||_{W^{1,\infty}(#2)}}
Let $\Omega \subset \R^n$ be open and bounded with a $C^1$ boundary. Let $V$ strictly contain $\Omega$.\footnote{The text says "$V$ strictly larger than $\Omega$"--this is the only interpretation I can make sense of.}

The task is to show that there exists a bounded linear operator 
\begin{equation}
E: W^{1,\infty}(\Omega) \to W^{1,\infty}(\R^n)
\end{equation}
satisfying, for all $w \in W^{1,\infty}(\Omega)$
\begin{enumerate}
\item $Eu = u$ for almost all $x\in\Omega$.
\item $\spt(Eu) \subset V$,
\item $\wnorm{Eu}{\R^n} \leq C\wnorm{u}{\Omega}$,
for some $C$ not depending on $u$.\\
\end{enumerate}

\textbf{Note}
I believe for the case $p = \infty$, the procedure outlined in Evans for $u \in C^1(\Omega)$ (pp 268-270) will more or less hold directly, so I will roughly follow the book. Where I feel it is obvious, I will then simply refer to the book for the sake of brevity.
\\
Now, let $u \in W^{1,\infty}$, and as in the book, assume $\partial \Omega$ is flat near $x_0$, lying in the plane $x_n = 0$.

Then there exists an open ball $B(x_0, r)$, which I split into
\begin{align*}
B^+ &:= B \cap \{x_n \geq 0\} \subset \overline{\Omega}\,,\\
B^- &:=
B \cap \{x_n \leq 0\} \subset R^n \setminus \overline{\Omega}\,.
\end{align*}
Set $\overline{u}$ to be
\begin{equation}
\overline{u}(x) =
\left\{
\begin{array}{lc}
u(x)\,,&x\in B^+\,,\\
-3u(x_1,x_2,\cdots,x_{n-1},-x_n)
+ 4u(x_1,x_2,\cdots,x_{n-1},-\frac{x_n}{2})
\,,
&x\in B^-\,.
\end{array}
\right.
\end{equation}
Moreover, define $\{v_j\}_{j=1}^n$ to be
\begin{equation}
v_j(x) =
\left\{
\begin{array}{lc}
u_{x_j}(x)\,,&x\in B^+\,,\\
3u_{x_j}(x_1,x_2,\cdots,x_{n-1},-x_n)
+ 4u_{x_j}(x_1,x_2,\cdots,x_{n-1},-\frac{x_n}{2})
\,,
&x\in B^-, j\neq n\,,
\\
3u_{x_n}(x_1,x_2,\cdots,x_{n-1},-x_n)
- 2u_{x_n}(x_1,x_2,\cdots,x_{n-1},-\frac{x_n}{2})
\,,
&x\in B^-\, j = n\,,
\end{array}
\right.
\end{equation}
for $j = 1, \cdots, n$.
Then the next step is to verify that $v_j$ is a weak derivative of $\overline{u}$ in $B$. Let $\phi \in C^\infty_c(B)$. 
For ease of notation, let $x' = x_1, x_2, \cdots, x_{n-1}$.
Then for $j = n$,
\begin{align*}
&\int_B \phi_{x_n} \overline{u}\dx
=
\int_{B^+}\!\! \phi_{x_n} \overline{u}\:\dx
+
\int_{B^-}\!\! \phi_{x_n} \overline{u}\:\dx
\\
&=
\int_{B^+}\!\! \phi_{x_n} u \:\dx\\
&+
\int_{B^+}\!\! \phi_{x_n}(x', -x_n) 
-3u(x',x_n)
+ 4u(x',\frac{x_n}{2}))
\:\dx \,.
\end{align*}
It is now safe to integrate by parts. Obtain
\begin{align*}
(\cdots)
=
&
\cancel{
\int_{x_n = 0}\!\!\!\! \phi u \:\mbox{D}s(x)
}
-
\int_{B^+}\!\! \phi u_{x_n} \:\dx\\
&+
\cancel{
\int_{x_n = 0}\!\!\!\! \phi
(-3u+ 4u)
 \:\mbox{D}s(x)
}
-
\int_{B^+}\!\! \phi(x',-x_n)(
3u_{x_n}(x',x_n) + 2u_{x_n}(x',\frac{x_n}{2})
) \:\dx \\
=&
-\int_{B^+}\!\! \phi u_{x_n} \:\dx
-\int_{B^-}\!\! \phi(3u_{x_n}(x',-x_n) - 2u_{x_n}(x',\frac{-x_n}{2}))\:\dx \\
&=
\int_B\!\!\phi v \:\dx\,.
\end{align*}
\newcommand{\hbnorm}[2]{||#1||_{C^{0,\gamma}(\overline{#2})}}
The cases $1 \leq j \leq n$ are similar and omitted. Hence, for any multiindex $\alpha$ with $|\alpha| = 1$, letting $D^\alpha u = v_\alpha$, we have $\int_B\!  D^\alpha \!\phi\: \overline{u}\:\dx
= \int_B D^\alpha \overline{u} \:\phi\: \dx$.

As in the book, it is clear that $\hbnorm{\overline{u}{B}} \leq C\hbnorm{u}{B^+}$.
\\

The rest of the proof now goes exactly as in the book, so the esteemed reader may skip the rest: if $\partial \Omega$ is not flat near $x_0$, use a $C^1$ homeomorphism to straighten it out, then exploit compactness of $\partial \Omega$ to cover it with a finite number, say $N$, of open sets $W_i$ in which to obtain extensions $u_i$ of $u$. Let $u_0 = u$, choose $W_0$ such that $\bigcup_{i=0}^N W_i = \Omega$, and let $\{\zeta_i\}_{i=0}^N$ form an associated partition of unity. Finally, let $v = \sum_{i=0}^N \zeta_i u_i$. Then $\hbnorm{v}{\Omega} \leq C\hnorm{u}$ for some $C > 0$, and we may define $E$ as the linear map mapping $u$ to $v$.
\qed

\section{Problem 6}
Let $u: \R^3 \to \R$ be given by
\begin{align}
&u(x) := |x - x_0|^{\alpha}\,,
&
x\in B(x_0;1) =: \Omega\,.
\end{align}
for some $\alpha > 0$. For $x \neq 0$, we have $u_{x_i} = -\alpha x_i |x - x_0|^{-\alpha-2}$, and so
\begin{equation}
|D u| = |\alpha| |x-x_0|^{-\alpha - 1}\,.
\end{equation}
For the notion of a weak derivative to make sense, we require that $\int_\Omega \phi_{x_i}\: u\:\dx = \int_\Omega u_{x_i} \:\phi :\dx$ for a test function $\phi$, $j = 1, 2, 3$. Let $0 < \epsilon < 1$, and compute
\begin{equation}
\int_{\Omega \setminus B(x_0,\epsilon}\!\!\!\!\!\!\!\!\!\!\!\!\!
\phi_{x_i} u
\:\dx
=
-
\int_{\Omega \setminus B(x_0,\epsilon}\!\!\!\!\!\!\!\!\!\!\!\!\!
\phi\: u_{x_i}
\:\dx
+ 
\int_{\partial B(x_0, \epsilon)}\!\!\!\!\!\!\!\!\!\!
\phi u \nu^i
\\mbox{d}S(x)\,,
\end{equation}
$\mathbf{\nu}$ denoting the inward pointing unit normal.

We have
\begin{equation}
\left|
\int_{\partial B(x_0, \epsilon)}\!\!\!\!\!\!\!\!\!\!
\phi u \nu^i
\mbox{d}S(x)
\right|
\leq
||\phi||_{L^\infty(\Omega)}
\int_{\partial B(x_0, \epsilon)}\!\!\!\!\!\!\!\!\!\!
\epsilon^{-\alpha}\:\mbox{d}S(x)
\leq
C \epsilon^{2-\alpha} \to 0
\end{equation}
as $\epsilon \to 0$ so long as $\alpha < 2$.

For a bound on $||Du||_{W^{1,2}(\Omega)}$, we have
\begin{equation}
\int_\Omega\!\!\! |Du|^2\:\dx
= 
\alpha^2
\int_\Omega\!\!\! |x-x_0|^{-2\alpha-2}\:\dx
= \alpha^2 4\pi
\int_0^1 \!\!\!  r^{-2\alpha}\:\mbox{d}r 
= 
\alpha^2 4\pi \frac{1}{1 - 2\alpha}
\end{equation}
which is finite iff $\alpha < 1/2$. This is consistent with the result in the book, pp. 260.\qed
\end{document}